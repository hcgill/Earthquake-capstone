\documentclass[]{article}
\usepackage{lmodern}
\usepackage{amssymb,amsmath}
\usepackage{ifxetex,ifluatex}
\usepackage{fixltx2e} % provides \textsubscript
\ifnum 0\ifxetex 1\fi\ifluatex 1\fi=0 % if pdftex
  \usepackage[T1]{fontenc}
  \usepackage[utf8]{inputenc}
\else % if luatex or xelatex
  \ifxetex
    \usepackage{mathspec}
  \else
    \usepackage{fontspec}
  \fi
  \defaultfontfeatures{Ligatures=TeX,Scale=MatchLowercase}
\fi
% use upquote if available, for straight quotes in verbatim environments
\IfFileExists{upquote.sty}{\usepackage{upquote}}{}
% use microtype if available
\IfFileExists{microtype.sty}{%
\usepackage{microtype}
\UseMicrotypeSet[protrusion]{basicmath} % disable protrusion for tt fonts
}{}
\usepackage[margin=1in]{geometry}
\usepackage{hyperref}
\hypersetup{unicode=true,
            pdftitle={Earthquake\_Capstone\_Report},
            pdfauthor={H. Ewton},
            pdfborder={0 0 0},
            breaklinks=true}
\urlstyle{same}  % don't use monospace font for urls
\usepackage{color}
\usepackage{fancyvrb}
\newcommand{\VerbBar}{|}
\newcommand{\VERB}{\Verb[commandchars=\\\{\}]}
\DefineVerbatimEnvironment{Highlighting}{Verbatim}{commandchars=\\\{\}}
% Add ',fontsize=\small' for more characters per line
\usepackage{framed}
\definecolor{shadecolor}{RGB}{248,248,248}
\newenvironment{Shaded}{\begin{snugshade}}{\end{snugshade}}
\newcommand{\AlertTok}[1]{\textcolor[rgb]{0.94,0.16,0.16}{#1}}
\newcommand{\AnnotationTok}[1]{\textcolor[rgb]{0.56,0.35,0.01}{\textbf{\textit{#1}}}}
\newcommand{\AttributeTok}[1]{\textcolor[rgb]{0.77,0.63,0.00}{#1}}
\newcommand{\BaseNTok}[1]{\textcolor[rgb]{0.00,0.00,0.81}{#1}}
\newcommand{\BuiltInTok}[1]{#1}
\newcommand{\CharTok}[1]{\textcolor[rgb]{0.31,0.60,0.02}{#1}}
\newcommand{\CommentTok}[1]{\textcolor[rgb]{0.56,0.35,0.01}{\textit{#1}}}
\newcommand{\CommentVarTok}[1]{\textcolor[rgb]{0.56,0.35,0.01}{\textbf{\textit{#1}}}}
\newcommand{\ConstantTok}[1]{\textcolor[rgb]{0.00,0.00,0.00}{#1}}
\newcommand{\ControlFlowTok}[1]{\textcolor[rgb]{0.13,0.29,0.53}{\textbf{#1}}}
\newcommand{\DataTypeTok}[1]{\textcolor[rgb]{0.13,0.29,0.53}{#1}}
\newcommand{\DecValTok}[1]{\textcolor[rgb]{0.00,0.00,0.81}{#1}}
\newcommand{\DocumentationTok}[1]{\textcolor[rgb]{0.56,0.35,0.01}{\textbf{\textit{#1}}}}
\newcommand{\ErrorTok}[1]{\textcolor[rgb]{0.64,0.00,0.00}{\textbf{#1}}}
\newcommand{\ExtensionTok}[1]{#1}
\newcommand{\FloatTok}[1]{\textcolor[rgb]{0.00,0.00,0.81}{#1}}
\newcommand{\FunctionTok}[1]{\textcolor[rgb]{0.00,0.00,0.00}{#1}}
\newcommand{\ImportTok}[1]{#1}
\newcommand{\InformationTok}[1]{\textcolor[rgb]{0.56,0.35,0.01}{\textbf{\textit{#1}}}}
\newcommand{\KeywordTok}[1]{\textcolor[rgb]{0.13,0.29,0.53}{\textbf{#1}}}
\newcommand{\NormalTok}[1]{#1}
\newcommand{\OperatorTok}[1]{\textcolor[rgb]{0.81,0.36,0.00}{\textbf{#1}}}
\newcommand{\OtherTok}[1]{\textcolor[rgb]{0.56,0.35,0.01}{#1}}
\newcommand{\PreprocessorTok}[1]{\textcolor[rgb]{0.56,0.35,0.01}{\textit{#1}}}
\newcommand{\RegionMarkerTok}[1]{#1}
\newcommand{\SpecialCharTok}[1]{\textcolor[rgb]{0.00,0.00,0.00}{#1}}
\newcommand{\SpecialStringTok}[1]{\textcolor[rgb]{0.31,0.60,0.02}{#1}}
\newcommand{\StringTok}[1]{\textcolor[rgb]{0.31,0.60,0.02}{#1}}
\newcommand{\VariableTok}[1]{\textcolor[rgb]{0.00,0.00,0.00}{#1}}
\newcommand{\VerbatimStringTok}[1]{\textcolor[rgb]{0.31,0.60,0.02}{#1}}
\newcommand{\WarningTok}[1]{\textcolor[rgb]{0.56,0.35,0.01}{\textbf{\textit{#1}}}}
\usepackage{graphicx,grffile}
\makeatletter
\def\maxwidth{\ifdim\Gin@nat@width>\linewidth\linewidth\else\Gin@nat@width\fi}
\def\maxheight{\ifdim\Gin@nat@height>\textheight\textheight\else\Gin@nat@height\fi}
\makeatother
% Scale images if necessary, so that they will not overflow the page
% margins by default, and it is still possible to overwrite the defaults
% using explicit options in \includegraphics[width, height, ...]{}
\setkeys{Gin}{width=\maxwidth,height=\maxheight,keepaspectratio}
\IfFileExists{parskip.sty}{%
\usepackage{parskip}
}{% else
\setlength{\parindent}{0pt}
\setlength{\parskip}{6pt plus 2pt minus 1pt}
}
\setlength{\emergencystretch}{3em}  % prevent overfull lines
\providecommand{\tightlist}{%
  \setlength{\itemsep}{0pt}\setlength{\parskip}{0pt}}
\setcounter{secnumdepth}{0}
% Redefines (sub)paragraphs to behave more like sections
\ifx\paragraph\undefined\else
\let\oldparagraph\paragraph
\renewcommand{\paragraph}[1]{\oldparagraph{#1}\mbox{}}
\fi
\ifx\subparagraph\undefined\else
\let\oldsubparagraph\subparagraph
\renewcommand{\subparagraph}[1]{\oldsubparagraph{#1}\mbox{}}
\fi

%%% Use protect on footnotes to avoid problems with footnotes in titles
\let\rmarkdownfootnote\footnote%
\def\footnote{\protect\rmarkdownfootnote}

%%% Change title format to be more compact
\usepackage{titling}

% Create subtitle command for use in maketitle
\newcommand{\subtitle}[1]{
  \posttitle{
    \begin{center}\large#1\end{center}
    }
}

\setlength{\droptitle}{-2em}

  \title{Earthquake\_Capstone\_Report}
    \pretitle{\vspace{\droptitle}\centering\huge}
  \posttitle{\par}
    \author{H. Ewton}
    \preauthor{\centering\large\emph}
  \postauthor{\par}
      \predate{\centering\large\emph}
  \postdate{\par}
    \date{9/28/2018}


\begin{document}
\maketitle

\#\#Data clean-up

The first step in cleaning up the code was to clean the file that was
downloaded from NOAA's Significant Earthquakes Database, found at
\url{https://www.ngdc.noaa.gov/nndc/struts/form?t=101650\&s=1\&d=1}.
This file, signif\_earthquakes presented several challenges. The file
needed to be reduced to only needed columns from the original columns
that were included in the original data set. Then, the data needed to be
filtered to include observations that included reliable magnitude
measurements. The chosen measurement scale for this analysis is the
Richter scale as it has been shown to be reliable and provides
substantial amounts of data. Therefore, all data prior to the creation
and regular use of the Richter scale (1935) was removed. Earthquakes
that were documented without a measurement were also filtered and
removed from the table. These processes left us with the table below:

\begin{verbatim}
## 
## Attaching package: 'dplyr'
\end{verbatim}

\begin{verbatim}
## The following objects are masked from 'package:stats':
## 
##     filter, lag
\end{verbatim}

\begin{verbatim}
## The following objects are masked from 'package:base':
## 
##     intersect, setdiff, setequal, union
\end{verbatim}

\begin{verbatim}
## 
## Attaching package: 'lubridate'
\end{verbatim}

\begin{verbatim}
## The following object is masked from 'package:base':
## 
##     date
\end{verbatim}

\begin{verbatim}
## Parsed with column specification:
## cols(
##   .default = col_integer(),
##   FLAG_TSUNAMI = col_character(),
##   SECOND = col_character(),
##   EQ_PRIMARY = col_double(),
##   EQ_MAG_MW = col_double(),
##   EQ_MAG_MS = col_double(),
##   EQ_MAG_MB = col_character(),
##   EQ_MAG_ML = col_double(),
##   EQ_MAG_MFA = col_character(),
##   EQ_MAG_UNK = col_double(),
##   COUNTRY = col_character(),
##   STATE = col_character(),
##   LOCATION_NAME = col_character(),
##   LATITUDE = col_double(),
##   LONGITUDE = col_double(),
##   MISSING = col_character(),
##   DAMAGE_MILLIONS_DOLLARS = col_character(),
##   TOTAL_MISSING = col_character(),
##   TOTAL_MISSING_DESCRIPTION = col_character(),
##   TOTAL_DAMAGE_MILLIONS_DOLLARS = col_character()
## )
\end{verbatim}

\begin{verbatim}
## See spec(...) for full column specifications.
\end{verbatim}

\begin{Shaded}
\begin{Highlighting}[]
\CommentTok{#Select relevant headings}
\NormalTok{signif_earthquakes <-}\StringTok{ }\KeywordTok{select}\NormalTok{(signif_earthquakes, }\StringTok{"YEAR"}\NormalTok{, }\StringTok{"MONTH"}\NormalTok{, }\StringTok{"DAY"}\NormalTok{, }\StringTok{"HOUR"}\NormalTok{, }\StringTok{"MINUTE"}\NormalTok{, }\StringTok{"SECOND"}\NormalTok{, }\StringTok{"FOCAL_DEPTH"}\NormalTok{, }\StringTok{"EQ_MAG_UNK"}\NormalTok{, }\StringTok{"COUNTRY"}\NormalTok{, }\StringTok{"LOCATION_NAME"}\NormalTok{, }\StringTok{"LATITUDE"}\NormalTok{, }\StringTok{"LONGITUDE"}\NormalTok{)}

\CommentTok{#Change headings to lowercase}
\NormalTok{signif_earthquakes <-}\StringTok{ }\KeywordTok{setNames}\NormalTok{(signif_earthquakes, }\KeywordTok{tolower}\NormalTok{(}\KeywordTok{names}\NormalTok{(signif_earthquakes)))}

\CommentTok{#Select data from 1935 to present}
\NormalTok{signif_earthquakes <-}\StringTok{ }\KeywordTok{filter}\NormalTok{(signif_earthquakes, year }\OperatorTok{>}\StringTok{ }\DecValTok{1934}\NormalTok{)}

\CommentTok{#Remove any magnitude values that are NA (indicates a recorded earthquake with no measurement)}
\NormalTok{signif_earthquakes}\OperatorTok{$}\NormalTok{eq_mag_unk[}\KeywordTok{is.na}\NormalTok{(signif_earthquakes}\OperatorTok{$}\NormalTok{eq_mag_unk)] <-}\StringTok{ }\DecValTok{0}
\NormalTok{signif_earthquakes <-}\StringTok{ }\KeywordTok{filter}\NormalTok{(signif_earthquakes, eq_mag_unk }\OperatorTok{>}\StringTok{ }\DecValTok{0}\NormalTok{)}

\KeywordTok{head}\NormalTok{(signif_earthquakes)}
\end{Highlighting}
\end{Shaded}

\begin{verbatim}
##   year month day hour minute second focal_depth eq_mag_unk country
## 1 1935     1   4   14     41   <NA>           7        6.2  TURKEY
## 2 1935     2  25    2     51   <NA>          NA        6.7  GREECE
## 3 1935     3   5   10     27   <NA>          NA        6.0    IRAN
## 4 1935     4  11   23     15   <NA>          14        6.3    IRAN
## 5 1935     5   1   10     24   <NA>          NA        6.1  TURKEY
## 6 1935     7  11   18     35   <NA>          10        6.3   JAPAN
##                       location_name latitude longitude
## 1                            TURKEY     40.5      27.5
## 2 GREECE:  NEAPOLIS-ANOGNIA (CRETE)     35.8      25.0
## 3                     IRAN:  ALBORZ     36.3      53.3
## 4       IRAN:  KEVSUT, ALBORZ, SARI     36.3      53.5
## 5                     TURKEY:  KIGI     39.3      40.6
## 6                    JAPAN:  HONSHU     35.0     138.0
\end{verbatim}


\end{document}
